\section{Real-Time Operating Systems (RTOS)}
\label{Real-Time Operating Systems (RTOS)}

\subsection{Operating System (OS) / Betriebssystem}

Ein Betriebssystem bildet die \textbf{Schnittstelle} zwischen den HW-Komponenten und der Anwendungssoftware des Benutzers.
Seine Aufgaben sind insbesondere:

\vspace{0.1cm}

\begin{minipage}[t]{0.58\columnwidth}
    \raggedright
    \begin{itemize}
        \item Benutzerkommunikation
        \item Laden, Ausführen, Unterbrechen und Beenden von Programmen
        \item \textbf{Verwaltung und Zuteilung der Prozessorzeit (Scheduling)}
    \end{itemize}
\end{minipage}
\hfill
\begin{minipage}[t]{0.38\columnwidth}
    \raggedright
    \begin{itemize}
        \item Speicherverwaltung
        \item Verwaltung und Betrieb der angeschlossenen Geräte
        \item Schutzfunktionen
    \end{itemize}
\end{minipage}


\subsection{RTOS / Echtzeitbetriebssystem}

Ein RTOS ist ein Betriebssystem, das \textbf{Echtzeitanforderungen} erfüllen kann.

\vspace{0.1cm}

\begin{outline}
    \1 Konzepte der POSIX-Programmierung finden hier Anwendung
    \1 Das \textbf{Prinzip} aller RTOS ist gleich
        \2 API und Umfang der RTOS sind unterschiedlich
\end{outline}


\para{Beispiele für RTOS}

\begin{minipage}[t]{0.45\columnwidth}
    \raggedright
    \begin{itemize}
        \item FreeRTOS von Amazon
        \item Zephyr von Linux Foundation
        \item VxWorks von Wind River Systems
        \item TI-RTOS von Texas Instruments
    \end{itemize}
\end{minipage}
\hfill
\begin{minipage}[t]{0.52\columnwidth}
    \raggedright
    \begin{itemize}
        \item \micro C/OS-III von Micrium
        \item QNX von Blackberry
        \item Windows 10 IoT Enterprise von Microsoft (nur soft real-time!)
    \end{itemize}
\end{minipage}



\subsection{FreeRTOS vs. Zephyr}

\scalebox{0.795}{
    \begin{tabular}{lll}
        \toprule
        \textbf{Kriterium}                                      & \textbf{FreeRTOS}                         & \textbf{Zephyr}                           \\
        \midrule
        \rowcolor{gray!40} \textbf{Leichtgewichtigkeit}         & Sehr leichtgewichtig                      & Ressourceninvensiver                      \\
        \textbf{Feature-Set}                                    & Grundlegende RTOS-Funktionalitäten        & Umfangreicher (Treiber, Netzwerk, FS)     \\
        \rowcolor{gray!40} \textbf{Hardware-Support}            & Sehr breit                                & Begrenzter, aber wachsender Support       \\
        \textbf{Lernkurve}                                      & Einfach und schnell                       & Anspruchsvoller, aber leistungsfähiger    \\
        \rowcolor{gray!40} \textbf{Einsatzgebiete}              & Kleinere, spezifische Projekte            & Skalierbare \& komplexere Anwendungen     \\
        \textbf{Lizenz}                                         & MIT                                       & Apache 2.0                                \\ 
        \bottomrule
    \end{tabular}
}


\subsubsection{Fazit der Gegenüberstellung}

\textrightarrow\ Welches RTOS geeigneter ist, \textbf{kommt auf die Anwendung an!} \\
Grundsätzlich gilt aber Folgendes:

\vspace{0.1cm}

\begin{outline}
    \1 \textbf{FreeRTOS} für Systeme mit begrenzten Ressourcen (\textbf{Flash $\bm{< 32 \, \kilo}$B, RAM $\bm{< 4 \, \kilo}$B} ideal) \\
        Bleibt performant für CPUs mit Taktfrequenz $\bm{< 100 \mega \hertz}$
        \2 Leichtgewichtiges RTOS für einfache / ressourcenbeschränkte Systeme
    \1  \textbf{Zephyr} wenn zusätzliche Funktionalität für komplexere Anwendungen unverzichtbar \\
        Bedarf an leistungsfähiger MCU (\textbf{Flash $\bm{\geq 64 \, \kilo}$B, RAM $\bm{\geq 8 \, \kilo}$B})
        \2 Modernes, skalierbares, funktionsreiches RTOS für komplexere Anwendungen und Projekte mit Netzwerk und Sicherheitsanforderungen
\end{outline}


\subsubsection{Beispiele für typische Anwendungen}

\scalebox{0.97}{
    \begin{tabular}{lll}
        \toprule
        \textbf{System}                                                         & \textbf{FreeRTOS}         & \textbf{Zephyr}       \\
        \midrule
        \rowcolor{gray!40} \textbf{Sensorsteuerung}                             & gut umsetzbar             & gut umsetzbar         \\
        \textbf{IoT-Gerät mit WLAN}                                             & gut umsetzbar             & gut umsetzbar         \\
        \rowcolor{gray!40} \textbf{Komplexeres IoT-Gerät (Bluetooth + FS)}      & schwierig umzusetzen      & gut umsetzbar         \\
        \textbf{Embedded Linux-Alternative (RTOS)}                              & nicht geeignet            & gut umsetzbar         \\
       \bottomrule
    \end{tabular}
}


