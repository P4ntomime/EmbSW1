\section{Patterns (Lösungsmuster)}

\textbf{Ein Software Pattern ist eine bekannte Lösung für eine Klasse von Problemen.}

\vspace{0.1cm}

\begin{minipage}[t]{0.48\columnwidth}
    \raggedright

    \begin{itemize}
        \item[+] Rad muss nicht immer neu erfunden werden 
        \item[+] Getestete / funktionierende Lösungen 
    \end{itemize}
\end{minipage}
\hfill
\begin{minipage}[t]{0.48\columnwidth}
    \raggedright

    \begin{itemize}
        \item[-] Wichtige Patterns müssen bekannt sein
        \item[-] Problemstellungen müssen als solche erkannt werden 
    \end{itemize}
\end{minipage}


\subsection{Arten von Patterns}

\begin{outline}
    \1 \textbf{Architekturmuster (Architectural Pattern)}
        \2 Legt die grundlegende Organisation einer Anwendung und die Interaktion zwischen den Komponenten fest
    \1 \textbf{Entwurfsmuster (Design Pattern)}
        \2 Die ursprüngliche Form des Pattern‐Ansatzes
    \1 \textbf{Implementationsmuster (Implementation Pattern)}
        \2 Behandelt grundsätzliche Implementationen immer wiederkehrender Codefragmente
\end{outline}


\subsection{Wichtige Patterns für Embedded Systems}

\subsubsection{Bereits bekannte Patterns}

\begin{minipage}[t]{0.48\columnwidth}
    \raggedright

    \begin{outline}
        \1 \textbf{FSM Implementationen}
            \2 State Pattern
            \2 Singleton Pattern
            \2 (Steuerkonstrukt mit switch-case)
            \2 (Tabellenvariante)
    \end{outline}
\end{minipage}
\hfill
\begin{minipage}[t]{0.5\columnwidth}
    \raggedright
    
    \begin{outline}
        \1 \textbf{'Mini-Patterns'}
            \2 Setzen / Löschen einzelner Bits
            \2 Behandlung asynchroner Ereignisse
                \3 Interrupts
                \3 Polling
    \end{outline}
\end{minipage}


\subsubsection{Creational Patterns}

Creational Patterns behandeln die \textbf{Erzeugung (und Vernichtung) von Objekten.}

\vspace{0.1cm}

% \begin{minipage}[t]{0.48\columnwidth}
%     \raggedright

%     \begin{outline}
%         \1 \textbf{Factory (Dependency injection)}
%             \2 Definition einer Schnittstelle zur Erzeugung eines Objekts, statt der direkten Erzeugung auf der Client-Seite
%         \1 \textbf{Singleton}
%             \2 stellt sicher, dass eine Klasse nur \textbf{ein einziges Objekt} besitzt
%     \end{outline}
% \end{minipage}
% \hfill
% \begin{minipage}[t]{0.5\columnwidth}
%     \raggedright
    
%     \begin{outline}
%         \1 \textbf{RAII (Resource Acquisition Is Initialization)}
%             \2 Die Belegung und Freigabe einer Ressource wird an die Lebensdauer eines Objektes gebunden. Dadurch wird eine
%                 Ressource z.B. 'automatisch' freigegeben.
%     \end{outline}
% \end{minipage}

\begin{outline}
    \1 \textbf{Factory (Dependency injection)}
        \2 Definition einer Schnittstelle zur Erzeugung eines Objekts, statt der direkten Erzeugung auf der Client-Seite
    \1 \textbf{Singleton}
        \2 stellt sicher, dass eine Klasse nur \textbf{ein einziges Objekt} besitzt
    \1 \textbf{RAII (Resource Acquisition Is Initialization)}
        \2 Die Belegung und Freigabe einer Ressource wird an die Lebensdauer eines Objektes gebunden. Dadurch wird eine Ressource 
            z.B. 'automatisch' freigegeben.
\end{outline}


\subsubsection{Structural Patterns}

Structural Patterns \textbf{vereinfachen Beziehungen} zu anderen Teilen.

\vspace{0.1cm}

% \begin{minipage}[t]{0.55\columnwidth}
%     \raggedright

%     \begin{outline}
%         \1 \textbf{Adapter (Wrapper, Translator, glue code)}
%             \2 Wandelt (adaptiert) eine Schnittstelle in eine für einen Client passendere Schnittstelle um
%         \1 \textbf{Facade}
%             \2 Bietet eine \textbf{einfache Schnittstelle} für die Nutzung einer meist viel \textbf{grösseren Library}
%     \end{outline}
% \end{minipage}
% \hfill
% \begin{minipage}[t]{0.43\columnwidth}
%     \raggedright
    
%     \begin{outline}
%         \1 \textbf{Proxy}
%             \2 'A proxy, in its most general form, is a class functioning as an interface to something else.'
%             \2 Oft ist es eine SW-Repräsentation eines HW-Teils, z.B. die Repräsentation einer Netzwerkverbindung
%     \end{outline}
% \end{minipage}

\begin{outline}
    \1 \textbf{Adapter (Wrapper, Translator, glue code)}
        \2 Wandelt (adaptiert) eine Schnittstelle in eine für einen Client passendere Schnittstelle um
    \1 \textbf{Facade}
        \2 Bietet eine \textbf{einfache Schnittstelle} für die Nutzung einer meist viel \textbf{grösseren Library}
    \1 \textbf{Proxy}
        \2 'A proxy, in its most general form, is a class functioning as an interface to something else.'
        \2 Oft ist es eine SW-Repräsentation eines HW-Teils, z.B. die Repräsentation einer Netzwerkverbindung
\end{outline}


\subsubsection{Behavioral Patterns}

Behavioral Patterns identifizieren \textbf{gemeinsame Kommunikationspatterns} zwischen Objekten und implementieren diese.

\vspace{0.1cm}

% \begin{minipage}[t]{0.48\columnwidth}
%     \raggedright

%     \begin{outline}
%         \1 \textbf{Mediator}
%             \2 definiert ein Objekt, welches das Zusammenspiel einer Menge von Objekten regelt
%             \2 ein Embedded System, das aus \textbf{mehreren Teilen} wie Sensoren und Aktoren besteht, wird \textbf{im Mediator
%                 softwaremässig zusammengebaut}
%     \end{outline}
% \end{minipage}
% \hfill
% \begin{minipage}[t]{0.48\columnwidth}
%     \raggedright
    
%     \begin{outline}
%         \1 \textbf{Observer (MVC)}
%             \2 Nicht nur bei Embedded Systems wichtig
%             \2 Wird als objektorientierte Variante präsentiert
%             \2 MVC-Prinzip kann auch prozedural mit Callbackfunktionen implementiert werden
%     \end{outline}
% \end{minipage}

\begin{outline}
    \1 \textbf{Mediator}
        \2 definiert ein Objekt, welches das Zusammenspiel einer Menge von Objekten regelt
        \2 ein Embedded System, das aus \textbf{mehreren Teilen} wie Sensoren und Aktoren besteht, wird \textbf{im Mediator
            softwaremässig zusammengebaut}
    \1 \textbf{Observer (MVC)}
        \2 Nicht nur bei Embedded Systems wichtig
        \2 Wird als objektorientierte Variante präsentiert
        \2 MVC-Prinzip kann auch prozedural mit Callbackfunktionen implementiert werden
\end{outline}

\vspace{0.1cm}

\textbf{Beispiel Mediator:} Bei einem Drucker mit mehreren Druckaufträgen von mehreren Personen teilt der Mediator die Aufträge jeweils korrekt

\columnbreak

\subsubsection{Concurrency Patterns}

Concurrency Patterns kümmern sich um die \textbf{Ausführung in multi-threaded Umgebungen.}

\vspace{0.1cm}

% \begin{minipage}[t]{0.48\columnwidth}
%     \raggedright

%     \begin{outline}
%         \1 \textbf{Active Object}
%             \2 entkoppelt den Methodenaufruf von der Methodenausführung \\
%                 Methode soll sich nicht kümmern, in welchem Kontext sie aufgerufen wird
%     \end{outline}
% \end{minipage}
% \hfill
% \begin{minipage}[t]{0.48\columnwidth}
%     \raggedright
    
%     \begin{outline}
%         \1 \textbf{Lock}
%             \2 Synchronisationsprimitive, welche den unteilbaren Zugriff read-modify-write implementiert
%         \1 \textbf{Monitor}
%             \2 Monitor versteckt Synchronisationsanforderungen vor Client
%     \end{outline}
% \end{minipage}

\begin{outline}
    \1 \textbf{Active Object}
        \2 entkoppelt den Methodenaufruf von der Methodenausführung \\
            Methode soll sich nicht kümmern, in welchem Kontext sie aufgerufen wird
    \1 \textbf{Lock}
        \2 Synchronisationsprimitive, welche den unteilbaren Zugriff read-modify-write implementiert
    \1 \textbf{Monitor}
        \2 Monitor versteckt Synchronisationsanforderungen vor Client
\end{outline}

