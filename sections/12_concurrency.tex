\section{Concurrency (Gleichzeitigkeit)}

Programme von praktischem Nutzen führen meist mehrere Arbeiten 'gleichzeitig' durch. Beispielsweise soll bei einem Embedded System ein Roboterarm 
bewegt werden, während 'gleichzeitig' mit einem übergeordneten System kommuniziert wird.


\subsection{Parallel Computing vs. Concurrent Computing}


\begin{outline}
    \1 \textbf{Parallel Computing}
        \2 Ausführung verschiedener Tasks \textbf{tatsächlich gleichzeitig}
        \2 Nicht möglich auf single-core System
    \1 \textbf{Concurrent Computing}
        \2 Ausführung verschiedener Tasks \textbf{wirkt nur gleichzeitig}
        \2 Verschiedene Tasks erhalten verschiedene 'time slices' \textrightarrow\ Ein Task pro time slice
        \2 Auf single- und multi-core Systemen möglich
\end{outline}


\subsection{Warum man Concurrency nicht verwenden sollte}

\begin{outline}
    \1 \textbf{Concurrency (mit Prozessen, Tasks, Threads) kostet immer}
        \2 Stack
        \2 Braucht context switch (Umschalten von einen zum anderen Prozess, Task, Thread) \\
            \textrightarrow\ Alter context (Registerwerte, Steck, etc.) muss gespeichert, neuer geladen werden 
        \2 Zugriff auf gemeinsame Ressourcen muss synchronisiert werden \\
            \textrightarrow\ fehleranfällig (wird vergessen / falsch gemacht)
    \1 \textbf{Komplexität steigt}
        \2 Sequenzielle Programme sind einfacher zu verstehen als parallele Programme
\end{outline}

\vspace{0.1cm}

\textbf{ \textrightarrow\ Concurrency nur dann einsetzen, wenn wirklich ein Nutzen vorhanden ist!}


\subsection{Synchronisation}

Wenn parallele Einheiten \textbf{gemeinsame Ressourcen} benützen, muss der Zugriff auf die Ressourcen geregelt (synchronisiert) werden.
Wenn dies nicht gemacht wird kann es sein, dass zwei Tasks dieselbe Ressource 'falsch' verwenden. \textrightarrow\ 'Deadlock'

\vspace{0.1cm}

\textbf{Achtung: } 'Ein Bisschen warten' ist \textbf{keine} Synchronisation!

\columnbreak

