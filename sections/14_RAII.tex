\section{Resource Acquisition Is Initialisation (RAII)}

RAII fordert sauber Speicher an und gibt diesen auch zuverlässig wieder frei. \\
\textrightarrow\ \textbf{Geht nur in C++}, nicht in Java
\textrightarrow\ Java hat einen garbage collector, der 'aufräumt', wann er will und nicht zwingend am Ende des scopes


\subsection{Grundkonzept von RAII}
\label{Grundkonzept von RAII}

\begin{outline}
    \1 Anforderung / Freigabe einer Ressource wird mit Hilfe einer \textbf{Klasse} implementiert
        \2 Konstruktor: Fordert Ressource an
        \2 Destruktor: Gibt Ressource wieder frei
    \1 Ressource kann wie ein \textbf{Objekt} behandelt werden
        \2 Sobald Objekt seine Gültigkeit verliert (z.B. out-of-scope), gibt Destruktor die Ressource 'automatisch' frei
\end{outline}


\subsection{RAII bei Heapobjekten}

\begin{minipage}[t]{0.45\columnwidth}
    \para{Problem}

    \lstinputlisting[aboveskip=1mm, belowskip=0mm, firstline=1, firstnumber=1, lastline=7,
                     morekeywords={[3], p}]
                    {snippets/RAII_heap.cpp}
\end{minipage}
\hfill
\begin{minipage}[t]{0.53\columnwidth}
    \para{Lösung}

    \lstinputlisting[aboveskip=1mm, belowskip=0mm, firstline=11, firstnumber=1, lastline=20,
                     morekeywords={[3], p}]
                    {snippets/RAII_heap.cpp}
\end{minipage}


\subsection{RAII bei Mutex}

\begin{minipage}[t]{0.56\columnwidth}
    \raggedright
    \para{Problem}
    
    Es muss sichergestellt werden, dass Mutex in jedem Fall wieder freigegeben wird

    \vspace{0.1cm}

    \begin{itemize}
        \item Exceptions können dazwischenkommen
        \item Freigabe muss auch bei vorzeitigem Verlassen mit \mylstbox{return} erfolgen
    \end{itemize}
\end{minipage}
\hfill
\begin{minipage}[t]{0.42\columnwidth}
    \lstinputlisting[aboveskip=0mm, belowskip=0mm, firstline=1, firstnumber=1, lastline=9,
                     morekeywords={[3], m}, morekeywords={[2], pthread_mutex_t}]
                    {snippets/RAII_mutex.cpp}
\end{minipage}


\para{Lösung}

\lstinputlisting[aboveskip=1mm, belowskip=1mm, firstline=12, firstnumber=1, lastline=20,
                 morekeywords={[3], mx, mutex, myMutex}, morekeywords={[2], pthread_mutex_t}]
                {snippets/RAII_mutex.cpp}

\lstinputlisting[aboveskip=1mm, belowskip=1mm, firstline=23, firstnumber=1, lastline=31,
                morekeywords={[3], lock, myMutex}, morekeywords={[2], ResourceLock}]
               {snippets/RAII_mutex.cpp}

\textbf{Hinweis:} Die Klasse \mylstbox{ResourceLock} ist '0 Byte gross, da die Funktionen 'inline' implementiert sind und eine Referenz
verwendet wurde


\subsubsection{'Auslagerung' einer Mutex}

Oft wird die Verwaltung (init/destroy) einer Mutex in eine Klasse \mylstbox{SharedResource} ausgelagert.
Die Klasse, welche die Mutex genötigt, beinhaltet ein privates Attribut vom Typ \mylstbox[morekeywords={[2], SharedResource}]{SharedResource}.


\para{Klasse SharedResource}

\lstinputlisting[aboveskip=1mm, belowskip=1mm,
                morekeywords={[3], mutex}, morekeywords={[2], pthread_mutex_t}]
               {snippets/RAII_SharedResource.cpp}

\textrightarrow\ \mylstbox{SharedResource} wird normalerweise zusammem mit \mylstbox{ResourceLock} verwendet!

