\section{Hardware Abstraction Layer (HAL)}
\label{Hardware Abstraction Layer (HAL)}

\subsection{Motivation für einen HAL}

In Programmen für Embedded Systems gibt es sehr häufig \textbf{Zugriffe auf Hardware-Register} (Setzen / Löschen von Bits.)
Würden diese Zugriffe auf HW-Register direkt bei Bedarf getätigt werden, würde sehr unleserlicher Code entstehen. Ausserdem wären Programme fehleranfällig.
Zudem muss der ganze Code geändert werden, falls die Zielhardware geändert wird.

\vspace{0.1cm}

Daher werden \textbf{Registerzugriffe im HAL abstrahiert}, was folgende Vorteile bietet:

\vspace{0.1cm}

\begin{outline}
    \1 Code bleibt leserlich und weniger fehleranfällig
    \1 Portierbarkeit auf andere Zielhardware sehr einfach
\end{outline}

\vspace{0.1cm}

\textbf{Hinweis:} Richtig implementiert verursacht die HAL keinen / kaum Overhead! \\
\textrightarrow\ HAL ist effizient!


\subsection{Organisation des HAL}

Der HAL wird in zwei Layer unterteilt:

\vspace{0.1cm}

\begin{description}
    \item[Board Support Library (BSL):] abstrahiert das PCB und \myul{muss} vom Hersteller des Boards zur Verfügung gestellt werden
    \item[Chip Support Library (CSL):] abstrahiert den Chip, (den \textmu C) und wird \myul{häufig} vom \textmu C Hersteller zur Verfügung gestellt  
\end{description}


\subsection{HAL in C}

So vieles wie möglich wird mittels \textbf{Inline-Funktionen} umgesetzt.

\vspace{0.1cm}

\begin{outline}
    \1 \textbf{Inline-Funktionen müssen in \myul{Headerfiles} definiert sein}, damit der Compiler auch Inlining machen kann
        \2 Inlining wird nur gemacht, wenn dem Compiler auch eine \textbf{Optimierungsstufe} mitgegeben wird
            z.B. \lstinline|clang -c -01 foo.c|
    \1 Wenn Funktionen \mylstbox{static} deklariert werden, wird garantiert, dass Funktionen nicht auch noch im Objectfile als 
        Funktion vorhanden sind 
    \1 Damit die \textbf{Namen eindeutig} sind, sollen Unitkürzel \mylstbox{csl}, bzw. \mylstbox{bsl} verwendet werden
        \2 Hinweis: In C++ viel eleganter dank Namespaces!
    \1 Port / Pin Deklarationen werden als \mylstbox{typedef (volatile) struct} im CSL umgesetzt
    \1 Im \lstinline|main.c| file braucht es \mylstbox{bls_init()}\textbf{Init-Funktionen}
        \2 Beispiel: \mylstbox[morekeywords={[3], statusIndicator, bsl_led1}]{bls_ledInit(&statusIndicator, bsl_led1);}
\end{outline}


\example{Ausschnitt aus einem CSL Header-File}

\lstinputlisting[aboveskip=0mm, belowskip=0mm,
                 morekeywords={[3], reg, bits, csl_bit0, csl_bit1, csl_bit2, csl_bit29, csl_bit30, csl_bit31}]
                {snippets/HAL_C_CSL_registers.h}


\para{Port Definitionen in CSL Header-File}

\begin{minipage}[t]{0.48\columnwidth}
    \lstinputlisting[aboveskip=1mm, belowskip=2mm,
                     firstline=1, lastline=12,
                     morekeywords={[3], ctrl, sel, mux, dir, pud, set, clear, toggle},
                     morekeywords={[2], csl_PortCtrlRegs, csl_PortDataRegs}]
                    {snippets/HAL_C_CSL.h}
\end{minipage}
\hfill
\begin{minipage}[t]{0.48\columnwidth}
    \lstinputlisting[aboveskip=1mm, belowskip=0mm,
                     firstline=14,, firstnumber=14, lastline=24,
                     morekeywords={[3], dat, ctrl, data, addrPortACtrlRegs, addrPortBCtrlRegs, addrPortADataRegs, addrPortBDataRegs},
                     morekeywords={[2], csl_PortCtrlRegs, csl_PortDataRegs, csl_Port}]
                    {snippets/HAL_C_CSL.h}
\end{minipage}

\lstinputlisting[aboveskip=1mm, belowskip=0mm,
                 firstline=26, , firstnumber=26, lastline=26,
                 morekeywords={[3], port, id, initSamplePeriod},
                 morekeywords={[2], csl_Port, csl_PortId}]
                {snippets/HAL_C_CSL.h}


\para{Pin Definitionen in CSL Header-File}

\lstinputlisting[aboveskip=1mm, belowskip=0mm,
                 firstline=31, firstnumber=28, lastline=38,
                 morekeywords={[3], port, bit, pin, id, mux, data, dat},
                 morekeywords={[2], csl_Port, csl_Pin, csl_PinId, csl_PinMux}]
                {snippets/HAL_C_CSL.h}


\example{Ausschnitt aus einem BSL Header-File}

In der Board Support Library wird die Verbindung der effektiv auf dem Board vorhandenen Komponenten mit dem \textmu C über die CSL abstrahiert.

\lstinputlisting[aboveskip=1mm, belowskip=0mm,
                 morekeywords={[3], pin, led}, morekeywords={[2], csl_Pin, bsl_Led}]
                {snippets/HAL_C_BSL.h}



\subsection{HAL in C++}

So vieles wie möglich wird mittels \textbf{Inline-Funktionen} umgesetzt.

\vspace{0.1cm}

\begin{outline}
    \1 \textbf{Inline-Funktionen müssen in \myul{Headerfiles} definiert sein}, damit der Compiler auch Inlining machen kann
        \2 Inlining wird nur gemacht, wenn dem Compiler auch eine \textbf{Optimierungsstufe} mitgegeben wird
            z.B. \lstinline|clang++ -c -01 foo.cpp|
    \1 Wenn Funktionen \mylstbox{static} deklariert werden, wird garantiert, dass Funktionen nicht auch noch im Objectfile als 
        Funktion vorhanden sind 
    \1 Damit die \textbf{Namen eindeutig} sind, sollen \textbf{Namespaces} \mylstbox{csl}, bzw. \mylstbox{bsl} definiert werden
    \1 Port / Pin Deklarationen werden als \mylstbox{class} im Namespace \mylstbox{csl} umgesetzt
    \1 Im \lstinline|main.cpp| file braucht es \textbf{keine Init-Funktionen mehr}
        \2 Die Initialisierung übernimmt der \textbf{Konstruktor}
\end{outline}


\example{Ausschnitt aus einem CSL Header-File}

\lstinputlisting[aboveskip=0mm, belowskip=0mm,
                 morekeywords={[3], reg, bits, bit0, bit1, bit2, bit29, bit30, bit31}]
                {snippets/HAL_CPP_CSL_registers.h}


\para{Port Definitionen in CSL Header-File}

\lstinputlisting[aboveskip=1mm, belowskip=0mm,
                 firstline=1, , firstnumber=1, lastline=29,
                 morekeywords={[3], ctrl, sel, mux, dir, pud, set, clear, toggle, dRegs, dRegs, id, initSamplePeriod, addrPortACtrlRegs, addrPortBCtrlRegs, addrPortADataRegs, addrPortBDataRegs,},
                 morekeywords={[2], DataRegs, CtrlRegs, Id}]
                {snippets/HAL_CPP_CSL.h}

% \columnbreak

\para{Pin Definitionen in CSL Header-File}

\lstinputlisting[aboveskip=1mm, belowskip=0mm,
                 firstline=33, , firstnumber=31, lastline=45,
                 morekeywords={[3], id, dir, mux, port, dat, bit, in, fun1},
                 morekeywords={[2], Port, Id, Direction, Mux}]
                {snippets/HAL_CPP_CSL.h}


\example{Ausschnitt aus einem BSL Header-File}

In der Board Support Library wird die Verbindung der effektiv auf dem Board vorhandenen Komponenten mit dem \textmu C über die CSL abstrahiert.

\lstinputlisting[aboveskip=1mm, belowskip=0mm,
                 morekeywords={[3], pin, id}, morekeywords={[2], Id, Pin}]
                {snippets/HAL_CPP_BSL.h}



\subsection{Client Program in C bzw. C++}

\begin{minipage}[t]{0.52\columnwidth}
    \para{Client Program in C}

    \lstinputlisting[xleftmargin=2.5ex, aboveskip=1mm, belowskip=0mm,
                     morekeywords={[3], statusIndicator, bsl_led1, statusSwitch, bsl_switch1}, morekeywords={[2], bsl_Led, bsl_Switch}]
                    {snippets/HAL_C_client_program.c}
\end{minipage}
\hfill
\begin{minipage}[t]{0.46\columnwidth}
    \para{Client Program in C++}

    \lstinputlisting[aboveskip=1mm, belowskip=0mm,
                     morekeywords={[3], led1, switch1, statusIndicator,statusSwitch}, morekeywords={[2], Led, Switch}]
                    {snippets/HAL_CPP_client_program.cpp}
\end{minipage}

