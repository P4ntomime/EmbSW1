\section{Interfacig in C}

\subsection{Plain Old Data Types (POD Types)}



\begin{minipage}[t]{0.58\columnwidth}
    \raggedright  
    POD Types sind Datentypen, welche bereits in C vorhanden sind. Sie funktionieren in C++ identisch wie in C.
\end{minipage}
\hfill
\begin{minipage}[t]{0.4\columnwidth}
    \raggedright  

    \begin{itemize}
        \item \mylstbox{char}, \mylstbox{short}, \mylstbox{int}, \mylstbox{long} \\
            \textrightarrow\ jeweils \mylstbox{signed} und \mylstbox{unsigned}
        \item \mylstbox{float}, \mylstbox{double}
    \end{itemize}
\end{minipage}


\subsection{Language Linkage}

\begin{itemize}
    \item \textbf{Linker benötigt einen eindeutigen Namen für jede Funktion}
    \item In Assembler muss ein eindeutiger Label entstehen, der mit BSR (Branch to subroutine) angesprungen wird
    \item \textbf{In C: einfach}, da Funktionsname projektweit eindeutig sein muss
    \item \textbf{In C++: komplizierter}, da Funktionen in einem Namespace liegen, zu einer Klasse gehören und auch noch überladen werden können
\end{itemize}


\subsubsection{C Language Linkage}

\begin{outline}
    \1 C Linker hängt für Interne Darstellung häufig ein \textbf{Underscore vorne an Funktionsnamen}, um ein Label in Assembler zu erhalten
        \2 \mylstbox{foo()} wird zu \mylstbox{_foo}
\end{outline}


\subsubsection{C++ Language Linkage}

\begin{outline}
    \1 C++ Linker verwendet für Interne Darstellung \textbf{Name Mangling} \\
        \textrightarrow\ \mylstbox{_<namespace>_<functionName>_<parameterTypesShortened>}
        \2 \mylstbox{foo(int)} wird zu \mylstbox{_foo_i}
        \2 \mylstbox{MyClass::foo(double, int)} wird zu \mylstbox{_MyClass_foo_d_i}
\end{outline}


\subsection{Festlegen der Language Linkage}

\subsubsection{Motivation}

\textbf{Aus C++} wird eine Funktion \mylstbox{foo(int)} aufgerufen, die in einer \textbf{kompilierten C-Bibliothek} vorliegt.

\vspace{0.1cm}

\begin{itemize}
    \item Der verwendete C++-Linker sucht nach einer Name Mangling Darstellung der Funktion: \mylstbox{_foo_i}
    \item Da es sich aber um eine C-Bibliothek handelt, heisst die Funktion aber \mylstbox{_foo}
\end{itemize}

\vspace{0.1cm}

\textrightarrow\ \textbf{Dem C++-Linker muss mitgeteilt werden, dass diese Funktion C Linkage hat!}


\subsection{Festlegen der Language Linkage -- Umsetungen C++ seitig}

Im \textbf{Funktionsprototypen} (C++-seitig) kann die Language Linkage festgelegt werden. Somit kann \textbf{C-Code aus C++ aufgerufen} werden.

\lstinputlisting[aboveskip=1mm, belowskip=1.5mm]{snippets/interfacing_linkage_prototypes.cpp}

\columnbreak

Optimalerweise wird die Language Linkage im \textbf{Header-File} vorgenommen. \\
\textbf{Auf folgende Weise versehene Header-Files können sowohl in C- als auch in C++-Dateien included werden.
Beide Compiler können diese Header übersetzen.}

\lstinputlisting[aboveskip=1mm, belowskip=1mm]{snippets/interfacing_linkage_header.h}

\textrightarrow\ \lstinline|__cplusplus| ist dann definiert, wenn mit einem C++ Compiler kompiliert wird!


\subsection{C++ Code aus C aufrufen}

C++ Elementfunktionen können \textbf{nicht direkt} aus C aufgerufen werden, da C weder Klassen, noch den \mylstbox{this} Pointer kennt.
Es werden \textbf{C Wrapper-Funktionen} benötigt, um C++ Code aufrufen zu können.


\example{Dining Philosopher}

\para{Ausgangssituation}

\begin{outline}
    \1 3 Personen an einem Tisch; 3 Stäbchen zum essen vorhanden
    \1 Jede Person braucht 2 Stäbchen zum essen
        \2 Stäbchen sollen mittels Threading weitergereicht werden
    \1 Ein \textbf{Deadlock} entsteht, wenn alle je ein Stäbchen haben und auf den nächsten warten
\end{outline}


\para{Philosopher Class}

\lstinputlisting[aboveskip=1mm, belowskip=0mm, firstnumber=1, firstline=1, lastline=17,
                 morekeywords={[3], attr, tid, p}, morekeywords={[2], pthread_t, pthread_attr_t}]
                {snippets/linkage_philosophers.cpp}


Die C Wrapper-Funktion \mylstbox{staticWrapper} muss \textbf{zwingend} \mylstbox{static} sein! 
Die Funktion muss auch ohne Zusammenhang mit einer Objekt-Instanz aufrufbar sein. \\
Damit in C der \mylstbox{this} Pointer zur Verfügung steht, muss dieser der Wrapper-Funktion als Parameter übergeben werden. 


\para{Philosopher::live() Funktion}

\lstinputlisting[aboveskip=1mm, belowskip=0mm, firstnumber=1, firstline=20, lastline=25,
                 morekeywords={[3], attr, tid}]
                {snippets/linkage_philosophers.cpp}


Philosopher wird erstellt ('spawned'), indem ein Thread erstellt wird. Als Argument der Thread Funktion \mylstbox{staticWrapper} 
wird der \mylstbox{this} Pointer übergeben.

