\section{Programming Style Guide}

Programme werden für \textbf{Programmierer} geschrieben, nicht für Compiler. Daher erleichtern Programmierkonventionen einen konsistenten Stil,
der auch von anderen (Programmierern) verstanden wird. \\
In der Praxis wird jedoch häufig Wert auf Unwichtiges (z.B. Namensgebung von Variablen) gelegt. Wichtiges wird jedoch oft übersehen.


\subsection{Grundsätzliche Konventionen}

\begin{outline}
    \1 Welche Regeln auch definiert werden: \textbf{Regeln \myul{konsistent und konsequent}} anwenden!
    \1 Mit \textbf{höchstem Warning level} kompilieren (\mylstbox{-Wall})
\end{outline}


\subsubsection{'Small stuff'}

\begin{outline}
    \1 Eine Anweisung pro Zeile, nur eine Variablendefinition pro Zeile
        \2 (nicht notwendig aber sehr sinnvoll)
    \1 Anweisungen in Blöcken einrücken (Empfehlung: 2 Leerzeichen)
        \2 Wo geschweifte Klammern stehen ist nicht wichig
\end{outline}


\subsection{Namenskonventionen}

\begin{outline}
    \1 Wenig underscores \mylstbox{_} verwenden
    \1 Keine Namen mit underscores \mylstbox{_} beginnen
    \1 Zusammengesetzte Namen mit \mylstbox{camelCase} betonen
    \1 Selbstdefinierte Typen / Klassen mit Grossbuchstaben beginnen
    \1 Funktionen, Methoden und Namespaces, Variablen und Objekte mit Kleinbuchstaben beginnen
    \1 Makros (\mylstbox{#define}) ausschliesslich mit Grossbuchstaben definieren, Trennung mit underscores \mylstbox{_} 
\end{outline}


\subsubsection{Namenskonventionen für Variablen}

\begin{outline}
    \1 Präfixe sehr zurückhaltend einsetzen (lieber vermeiden)
    \1 Schleifenvariablen mit Namen \mylstbox{i, j} vom Typ \mylstbox{size_t}
    \1 \mylstbox{c, ch} für \mylstbox{char} Variablen
    \1 \mylstbox{s, str} für \mylstbox{string} bzw. \mylstbox{char*} Variablen
\end{outline}


\subsection{Typen mit genauer Breite}

\begin{outline}
    \1 Ab C99 werden im Headerfile \mylstbox{<stdint.h>} verschiedene Typen mit genauer Breite definiert. 
        \2 signed: \mylstbox{int8_t}, \mylstbox{int16_t}, \mylstbox{int32_t}
        \2 unsigned: \mylstbox{uint8_t}, \mylstbox{uint16_t}, \mylstbox{uint32_t}
\end{outline}

